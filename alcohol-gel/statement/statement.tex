\documentclass{oci}
\usepackage[utf8]{inputenc}
\usepackage{lipsum}

\title{Alcohol Gel}

\begin{document}
\begin{problemDescription}
Sebastián siempre tiene un negocio en mente.
El año pasado su idea fue hacerse rico vendiendo alcohol gel.
Lamentablemente su negocio fue un fracaso, y ahora tiene problemas para convencer inversionistas de que
inviertan en su siguiente emprendimiento.
Afortunadamente, Sebastián tuvo una brillante idea que cree le ayudará a ocultar su reciente fracaso.
Si bien el balance total de la venta de alcohol gel fue negativo, si logra encontrar el
periodo de tiempo en que el negocio tuvo el mejor rendimiento, podría mostrar solo esa parte de los
datos a sus potenciales inversores.

Sebastián compró y vendió alcohol gel durante $n$ días numerados de 1 a $n$.
Para cada uno de esos días cuenta con los siguientes datos:
\begin{itemize}
	\item La cantidad $c_i$ de paquetes de alcohol gel que compró el día $i$.
	\item El precio $p_i$ al que compró cada paquete el día $i$.
	\item La cantidad $k_i$ de paquetes que logró vender el día $i$.
	\item El precio $v_i$ al que vendió cada paquete el día $i$.
\end{itemize}
Notar que los paquetes que no son vendidos en un día pueden ser vendidos en los días siguientes y por
lo tanto para un día $i$ el valor $k_i$ puede ser mayor que $c_i$.

El balance $b_i$ del día $i$ se define como la resta entre la cantidad de dinero que Sebastián
recibió ese día producto de sus ventas y la cantidad de dinero que Sebastián gastó en comprar alcohol gel.
Específicamente, $b_i = k_i\times v_i - c_i\times p_i$.
Por ejemplo, si en el día $1$ ocurre que $c_1 = 2, p_1 = 6, k_1=1, v_1=5$, entonces el balance $b_1$
para el día $1$ es $(1 \times 5) - (2 \times 6) = -7$.
El balance de un periodo de días consecutivos corresponde a la suma de los balances de esos días.
Por ejemplo, continuando con el ejemplo anterior, si en el día $2$ ocurre que
$c_2 = 3, p_2 = 3, k_2=4, v_2=7$ entonces el balance del día 2 es $b_2=19$, y por tanto del balance
del periodo comprendido entre los días $1$ y $2$ (incluidos) es $(-7) + 19 = 12$.

Dados los datos correspondientes a los $n$ días, tu tarea es ayudar a Sebastián a identificar los días
$j$ y $k$ tales que el periodo comprendido entre los días $j$ y $k$ tenga ganancia máxima.
Es decir, el periodo cuyo balance sea positivo y que además su valor sea máximo.
Si hay múltiples periodos con ganancia máxima, deberás identificar el de mayor largo.
Si hay múltiples periodos con igual ganancia máxima e igual largo máximo entonces cualquiera de ellos
es una respuesta válida.
En caso de que ningún periodo tenga un balance positivo, la respuesta deberá ser \verb|IMPOSIBLE|.

\end{problemDescription}

\begin{inputDescription}
La primera línea contiene un entero $n$ ($0 < n \leq 10^5$) correspondiente al número de días.
Las siguientes $n$ líneas describen cada uno de los días.
La línea $i$-ésima contiene los enteros $c_i, p_i, k_i$ y $v_i$ ($0 < c_i, p_i, k_i, v_i \leq 100$)
correspondientes al día $i$.
Se garantiza además que la cantidad $v_i$ de paquetes vendidos es siempre menor o igual
que la cantidad $c_i$ de paquetes comprados el día $i$ más la cantidad de paquetes no vendidos
en días anteriores, es decir, Sebastián solo puede vender lo que tiene en stock.
\end{inputDescription}

\begin{outputDescription}
En caso de haber solución la salida debe contener tres enteros $j$, $k$ y $g$.
Los enteros $j$ y $k$ representan respectivamente el día de inicio y de fin del periodo con ganancia
y largo máximo.
El valor $g$ corresponde a la ganancia en el periodo comprendido entre los días $j$ y $k$.
En caso de no haber ningún período de balance positivo deberás imprimir \verb|IMPOSIBLE|.
\end{outputDescription}

\begin{scoreDescription}
  \subtask{20}
  Se probarán varios casos donde $1 \leq n \leq 100$.
  \subtask{30}
  Se probarán varios casos donde $1 \leq n \leq 5000$.
  \subtask{50}
  Se probarán varios casos sin restricciones adicionales.
\end{scoreDescription}

\begin{sampleDescription}
\sampleIO{sample-1}
\sampleIO{sample-2}
\sampleIO{sample-3}
\end{sampleDescription}

\end{document}
