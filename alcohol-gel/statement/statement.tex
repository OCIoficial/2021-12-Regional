\documentclass{oci}
\usepackage[utf8]{inputenc}
\usepackage{lipsum}

\title{Alcohol Gel}

\begin{document}
\begin{problemDescription}
Sebastían siempre tiene un negocio en mente. El año pasado su idea fue hacerse rico vendiendo alcohol gel al por mayor. Lamentablemente no le fue bien, y ahora no sabe cómo convencer financistas para que inviertan en su nuevo emprendimiento.
 Si bien el balance total de su negocio fue negativo, Sebastían tuvo la siguiente idea: encontrar el período de tiempo en que su negocio funcionó mejor, y mostrar solo esa parte de los datos a sus potenciales inversores. 
Sebastían compró y vendió alcohol gel durante $n$ días, y para cada uno de esos días cuenta con los siguientes datos:
\begin{itemize}
	\item La cantidad $c_i$ de paquetes de alcohol que compró. 
	\item El precio $p_i$ al que compró cada paquete ese día.
	\item La cantidad $k_i$ de paquetes que logró vender ese día.
	\item El precio $v_i$ al que vendío cada paquete ese día.
\end{itemize}

La ganancia de un día se define como la resta entre la cantidad de dinero que Sebastían recibió ese día producto de sus ventas y la cantidad de dinero que Sebastían gastó al comprar. Por ejemplo, si en el día $1$ ocurre que $c_1 = 2, p_1 = 6, k_1=1, v_1=5$, entonces su ganancia fue $(1 \cdot 5) - (2 \cdot 6) = -7$.
 La ganancia de una secuencia de días corresponde a la suma de las ganancias de esos días. Por ejemplo, si en el día $2$ ocurre que $c_2 = 3, p_2 = 3, k_2=4, v_2=7$, entonces la ganancia del día $2$ es $19$, y por tanto la ganancía del período entre los días $1$ y $2$ (incluidos) es $(-7) + 19 = 12$.

Dados los datos correspondientes a $n$ días, deberás ayudar a Sebastián a identificar los días $a$ y $b$ tales que la ganancía del período entre los días $a$ y $b$ es lo más alta posible. Si hay multiples períodos con la ganancia máxima entonces cualquiera de ellos es una respuesta válida. En caso de que ningún período tenga una ganancia positiva, la respuesta deberá ser "IMPOSIBLE".

\end{problemDescription}

\begin{inputDescription}
La primera línea $n$ indica el número de días. A continuación la siguen $n$ líneas, cada una de la cuales está compuesta por $4$ enteros no negativos, $c_i, p_i, k_i$ y $v_i$, para $i$ entre $1$ y $n$. Se garantiza además que la cantidad vendida en un día $i$ 
es siempre mayor o igual que la cantidad comprada hasta ese día menos la cantidad vendida hasta el día anterior.
\end{inputDescription}

\begin{outputDescription}
Deberás imprimir $3$ enteros separados por espacios: $a \: b \: g$, donde $a$ y $b$ representan el día en que comienza y el día en termina el período de máxima ganancia, respectivamente, y $g$ representa la ganancia obtenida en el período entre $a$ y $b$ incluidos. En caso de no haber ningún período de ganancia positiva, deberás imprimir la palabra "IMPOSIBLE".
\end{outputDescription}

\begin{scoreDescription}
  \subtask{20}
  $1 \leq n \leq 100, \; \; 1 \leq c_i, k_i, p_i, v_i \leq 100$ 
  \subtask{30}
  $1 \leq n \leq 5000, \; \; 1 \leq c_i, k_i, p_i, v_i \leq 100$ 
  \subtask{50}
  $1 \leq n \leq 10^5, \; \; 1 \leq c_i, k_i, p_i, v_i \leq 100$ 
\end{scoreDescription}

\begin{sampleDescription}
\sampleIO{sample-1}
\sampleIO{sample-2}
\sampleIO{sample-3}
\end{sampleDescription}

\end{document}
