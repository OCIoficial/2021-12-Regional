\documentclass{oci}
\usepackage[utf8]{inputenc}
\usepackage{lipsum}

\title{Ejemplo}

\begin{document}
\begin{problemDescription}
  Alicia y Bob están jugando un juego muy particular en un calendario de un mes de 30 días.

  El juego consiste en lo siguiente, comenzando en la fecha $x$ del mes:
  \begin{enumerate}
    \item Si el día actual está tachado, terminan de jugar.
    \item Alicia y Bob tachan el día actual $x$.
    \item Si el día actual ($x$) corresponde a:
    \begin{itemize}
      \item[Lunes:] Avanzan de uno en uno hasta encontrar un día no tachado.
      \item[Martes:]  El siguiente día será el reflejo del día actual con respecto al centro del mes. Por ejemplo:
      \begin{itemize}
        \item Si $x=1$, el siguiente día será $30$.
        \item Si $x=2$, el siguiente día será $29$.
        \item Si $x=15$, el siguiente día será $16$.
        \item Si $x=30$, el siguiente día será $1$.
      \end{itemize}
      \item[Miércoles:] Si $x$ es par, avanzan un día. Si no, retroceden un día.
      \item[Jueves:] Avanzan $10$ días.
      \item[Viernes:] Si $x$ es par, el siguiente día será $\frac{x}{2}$. Si no, será $3 \cdot x + 1$.
      \item[Sábado:] El siguiente día será $2 \cdot x$.
      \item[Domingo:] Avanzan de dos en dos hasta encontrar un día no tachado.
    \end{itemize}
  \end{enumerate}

  \textbf{Importante:} Si en cualquier momento se pasan del día 30, vuelven al día 1. Por ejemplo: si caen en el día 35 volverán al día 5, y así.

\end{problemDescription}

\begin{inputDescription}
  La primera y única línea de la entrada contiene dos números enteros separados por espacios:
  \begin{itemize}
    \item $d$ ($0 \leq d \leq 6$) indicando a qué día corresponde el primer día del mes. Si $d$ es $0$, el mes comienza un lunes, si es $1$ un martes y así hasta domingo.
    \item $x$ ($1 \leq x \leq 30$) indicando el día inicial del juego.
  \end{itemize}
\end{inputDescription}

\begin{outputDescription}
  Debes imprimir un solo número entero indicando el día del mes en el que terminan el juego.
\end{outputDescription}

\begin{scoreDescription}
  \subtask{40}
  $d = 1$
  \subtask{60}
  $1 \leq d \leq 7$
\end{scoreDescription}

\begin{sampleDescription}
\sampleIO{sample-1}
\sampleIO{sample-2}
\end{sampleDescription}

\end{document}
