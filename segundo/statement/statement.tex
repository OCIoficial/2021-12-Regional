\documentclass{oci}
\usepackage[utf8]{inputenc}
\usepackage{lipsum}
\usepackage{xspace}

\newcommand{\caballo}{El Pituco\xspace}

\title{Segundo}

\begin{document}
\begin{problemDescription}
Sergio es un fanático de las apuestas.
Cada semana va al hipódromo para levantar apuestas sobre los resultados de las carreras.
Hasta ahora no ha tenido mucha suerte, pero tras años de estudios cree que encontró la fórmula
infalible para hacerse millonario.

Su plan es entrar en una de las apuestas más riesgosas, pero a su vez una de las que más paga.
La apuesta consiste en adivinar el tiempo exacto en que un caballo terminará la carrera.
Esto parece imposible, pero Sergio tiene la información de todas las carreras pasadas de \caballo,
su caballo favorito, con la cual está seguro que puede predecir el resultado.

Específicamente, para cada carrera pasada, Sergio tiene anotado en su libreta el tiempo en que
\caballo terminó la carrera.
Sergio sabe que \caballo está en una racha y por lo tanto obtendrá un buen tiempo.
Después de mucho pensarlo, también determinó que es poco probable que alcance su mejor tiempo entonces
se decidió por apostar por el segundo mejor tiempo.
?`Podrías ayudar a Sergio a encontrar este valor?
\end{problemDescription}

\begin{inputDescription}
La entrada del problema consiste en dos líneas.
La primera línea contiene un entero $n$ $(2 \le n \le 10^6)$: el número de carreras pasadas para
las cuales Sergio tiene información.
Las segunda línea contiene $n$ enteros.
El $i$-ésimo entero $t_i$ ($0 < t_i \leq 10^9$) indica el tiempo en milisegundos en que \caballo
terminó la carrera $i$-ésima.
Se garantiza que \emph{no} todos los valores $t_i$ serán iguales, es decir, hay al menos un valor
que es distinto a los demás.
\end{inputDescription}

\begin{outputDescription}
La salida debe contener un entero correspondiente al segundo mejor tiempo en que \caballo
terminó una carrera pasada.
\end{outputDescription}

\begin{scoreDescription}
  \subtask{10}
  Se probarán varios casos dónde $n \le 3$.
  \subtask{30}
  Se probarán varios casos dónde $t_i\ne t_j$ para todo $1 \leq i, j \leq n$,
  es decir, la entrada no contiene valores repetidos.
  \subtask{60}
  Se probarán varios casos sin restricciones adicionales.
\end{scoreDescription}

\begin{sampleDescription}
\sampleIO{sample-1}
\sampleIO{sample-2}
\end{sampleDescription}

\end{document}
