\documentclass{oci}
\usepackage[utf8]{inputenc}
\usepackage{lipsum}

\title{Segundo}

\begin{document}
\begin{problemDescription}
Sergio es un fanático de las apuestas.
Va al hipódromo una vez al mes a participar en las apuestas para levantar apuestas sobre cuál será el caballo más rápido.

Tras años de experiencia, Sergio asegura que existe un patrón para adivinar el caballo que ganará la carrera que viene a continuación:
El caballo que gana cada carrera es el mismo caballo que salió en segundo lugar en la misma carrera del año pasado.

Usando este dato, ¿puedes ayudar a Sergio a determinar cuál es el caballo que ganaría según este patrón?
\end{problemDescription}

\begin{inputDescription}
La entrada del problema consiste en dos líneas.
La primera línea contiene un entero $n$ $(2 \le n \le 10^6)$: el número de caballos que compitieron en la misma carrera del año pasado.
La segunda línea contiene $n$ enteros. El $i$-ésimo entero $t_i$ $(1 \le t_i \le 10^9)$ corresponde al tiempo de llegada a la meta, en milésimas de segundo, del caballo $i$ durante esa carrera. Los números de los caballos siempre son los números entre $0$ y $n-1$, sin que dos caballos tengan el mismo número.
Puede asumir que todos los caballos del año pasado corren este año.
\end{inputDescription}

\begin{outputDescription}
La salida debe contener una línea con un único entero $i$: el número del caballo que ganará la carrera correspondiente a este año.
\end{outputDescription}

\begin{scoreDescription}
  \subtask{10}
  $n \le 3$
  \subtask{40}
  $n \le 10^5$
  \subtask{50}
  No hay restricciones adicionales.
\end{scoreDescription}

\begin{sampleDescription}
\sampleIO{sample-1}
\sampleIO{sample-2}
\end{sampleDescription}

\end{document}
