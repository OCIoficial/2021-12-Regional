\documentclass{oci}
\usepackage[utf8]{inputenc}
\usepackage{lipsum}

\title{Ejemplo}

\begin{document}
\begin{problemDescription}

La nación de Nlogonia está cubierta completamente de montañas.
Por esta razón, se ha convertido en un popular destino para los fanáticos de los planeadores
flexibles ultraligeros, también conocidos popularmente como parapentes.
Debido a la multitud de montañas, Nlogonia presenta a los pilotos de parapente la posibilidad de
elegir entre una inmensidad de \emph{rutas} diferentes.
Para escoger una nueva ruta, los pilotos solo deben elegir la cima de una montaña desde donde
partir y una montaña de menor altura a la cual descender.

La geografía de Nlogonia es también peculiar pues sus montañas están dispuestas en una perfecta
grilla de $M$ filas y $N$ columnas.
Las filas en la grilla son numeradas de 1 a $M$ y avanzan en la dirección sur.
Las columnas están numeradas de 1 a $N$ y avanzan en la dirección este.
Decimos que la montaña en la fila $i$ y columna $j$ se encuentra en la posición $(i,j)$ y que
tiene altura $A(i,j)$.
Por lo tanto, podemos representar una ruta con dos pares $(i, j)$ y $(k,l)$ donde el primer par
corresponde a la posición de la montaña de inicio y el segundo par a la posición de la montaña
final.
La \emph{altura} de una ruta corresponde a la diferencia de alturas entre las montañas de
inicio y final ($A(i, j) - A(k, l)$).

Maiki, una experimentada piloto, quiere batir el récord del mayor descenso en parapente.
Naturalmente, Nlogonia presenta el lugar adecuado para encontrar la ruta perfecta.
Maiki sabe que para poder siquiera acercarse al récord debe elegir una ruta donde la dirección
del viento le favorezca.
El viento en Nlogonia sopla siempre en dirección sureste.
Por lo tanto, Maiki está interesada en encontrar una ruta cuya altura sea máxima,
pero solo entre las rutas válidas que vayan en la dirección sureste.
Específicamente, una ruta es válida si las posiciones de inicio $(i, j)$ y final $(k, l)$
cumplen $i < k$ y $j < l$.

DIBUJO[marcar varias rutas y decir cuales son validas y cuales no]

Maiki es una piloto experto, pero la programación no es lo suyo y necesita de tu ayuda para
encontrar la ruta válida de mayor altura.

\end{problemDescription}

\begin{inputDescription}
La primera línea de la entrada contiene dos enteros $M$ y $N$ ($0 < M \leq 1000$, $0 < N \leq 1000$)
correspondientes respectivamente a la cantidad de filas y columnas en la grilla.
Cada una de las siguientes $M$ líneas contienen $N$ enteros y describe una fila.
El entero $j$-ésimo de la fila $i$-ésima corresponde a la altura de la montaña en la posición $(i,j)$.
\end{inputDescription}

\begin{outputDescription}
La salida debe contener un único entero correspondiente a la altura máxima.
\end{outputDescription}

\begin{scoreDescription}
  \subtask{20}
  Descripción Subtarea1
  \subtask{30}
  Descripción Subtarea2
  \subtask{50}
  Descripción Subtarea3
\end{scoreDescription}

\begin{sampleDescription}
\sampleIO{sample-1}
\sampleIO{sample-2}
\end{sampleDescription}

\end{document}
